

\ifthenelse{\equal{\Degree}{undergraduate}}
{
	% TOC entry format
	% % 目录中章标题字体为黑体
	% \renewcommand{\cftchapfont}         {\heiti \zihao{-4}}

	\renewcommand{\cftchapnumwidth}       {1.0em}
	\renewcommand{\cftsecnumwidth}        {1.0em}
	\renewcommand{\cftsubsecnumwidth}     {1.0em}
	\renewcommand{\cftsubsubsecnumwidth}  {1.0em}

    \fancypagestyle{previous}
    {
        \commonhead{}
        \fancyfoot{}
    }
    \renewcommand{\prevstyle}
    {
        \cleardoublepage{}
        \assignpagestyle{\chapter}{previous}
        \thispagestyle{previous}
        \pagestyle{previous}
        \pagenumbering{Roman}
        \resetpagecounter{}
        \numberingstyle{}
    }

    \newcommand{\PaperContentIndent}{
        \setlength{\cftsecindent}       {0em}
        \setlength{\cftsubsecindent}    {2em}
        \setlength{\cftsubsubsecindent} {4em}
    }

    \newcommand{\ProposalContentIndent}{
        \setlength{\cftsecindent}       {2em}
        \setlength{\cftsubsecindent}    {4em}
        \setlength{\cftsubsubsecindent} {6em}
    }

    

    \setenumerate{
        itemsep=0pt,
        partopsep=0pt,
        parsep=\parskip,
        topsep=0pt,
        wide=0em,
        labelindent=\parindent,
        listparindent  = \parindent,
    }

    \setitemize{
        itemsep=0pt,
        partopsep=0pt,
        parsep=\parskip,
        topsep=0pt,
        wide=0em,
        labelindent=\parindent,
        listparindent  = \parindent,
    }

    \setdescription{
        itemsep=0pt,
        partopsep=0pt,
        parsep=\parskip,
        topsep=0pt,
        wide=0em,
        labelindent=\parindent,
        listparindent  = \parindent,
    }

    \setenumerate[1]{label=(\arabic*)}
    \setenumerate[2]{label=\arabic*)}

    \newcommand{\tableformat}{
        \appto{\@floatboxreset}{\fangsong \zihao{5}} % 设置字体为仿宋,字号为 5 号
        \appto{\@floatboxreset}{\renewcommand{\arraystretch}{1.1}} % 设置行距为原来的 1 倍
        \appto{\@floatboxreset}{\renewcommand{\tabcolsep}{1pt}}
    }
}{}

