\ifthenelse{\equal{\Degree}{undergraduate}}
{
    \ifthenelse{\equal{\Type}{thesis}}
    {
        \renewcommand{\TitleTypeName}{浙江大学本科生毕业论文(设计)}
    }{}

    \newcommand{\PaperContentIndent}{
        \setlength{\cftsecindent}       {0em}
        \setlength{\cftsubsecindent}    {2em}
        \setlength{\cftsubsubsecindent} {4em}
    }

    \newcommand{\ProposalContentIndent}{
        \setlength{\cftsecindent}       {2em}
        \setlength{\cftsubsecindent}    {4em}
        \setlength{\cftsubsubsecindent} {6em}
    }


    % 用于生成翻译部分的开头内容
    \newcommand{\translationheader}[4]
    {
        {
            {
                % 设置粗体系列
                \bfseries
                % 设置段落缩进为0
                \setlength{\parindent}{0pt}
                % 输出译文题目
                \par \zihao{-3}译文题目:  {#1}
                % 输出原文题目,引用文献中的标题
                \par \zihao{-3}原文题目: \citefield{#2}{title}
                % 输出原文来源,完整引用文献
                \par \zihao{-3}原文来源: \fullcite{#2}
                % 引用文献
                \nocite{#2}
            }
            {
                % 换行
                \newline
                % 输出摘要
                \par \textbf{摘要:}  {#3}
                % 输出关键字
                \par \textbf{关键字:} {#4}
            }
        }
    }

    \newcommand{\hiddenTocSucsec}{
        \changelocaltocdepth{0}
    }

    \newcommand{\showTocSubsec}{
        % 根据 general/layout.tex 的设置,目录深度为 2
        \changelocaltocdepth{2}
    }

    \newcommand{\AddToContentWithoutPagenum}[2]{
        \addtocontents{toc}{\cftpagenumbersoff{#1}} % 页码不显示
        \phantomsection
        \addcontentsline{toc}{#1}{#2}
        \addtocontents{toc}{\cftpagenumberson{#1}} % 页码显示
    }
    
    \newcommand{\AddPagesTitleWithPunctuation}[3]{
        \cleardoublepage
        \AddToContentWithoutPagenum{#1}{《#2》}
        \begin{center}
            {#3  #2}
        \end{center}
    }
    
    \newcommand{\AddPagesNoTitleWithPunctuation}[2]{
        \cleardoublepage
        \AddToContentWithoutPagenum{#1}{《#2》}
    }

	\newcommand{\signatureWithYear}[5]
	{
		\begin{flushright}
			#1 \uline{\quad #2 \quad} \\
			#3 \quad 年 \quad #4 \quad 月 \quad #5 \quad 日
		\end{flushright}
	}

    \newcommand{\colwidth}[1]{\dimexpr#1-2\tabcolsep\relax}
}{}