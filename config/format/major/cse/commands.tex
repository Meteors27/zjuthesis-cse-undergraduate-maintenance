\ifthenelse{\equal{\Degree}{undergraduate}}
{
    % undergraduate
    \ifthenelse{\equal{\Type}{thesis}}
    {
        \renewcommand{\TitleTypeName}{浙江大学本科生毕业论文(设计)}
    }{}
}{}

% 用于生成翻译部分的开头内容
\newcommand{\translationheader}[4]
{
    {
        {
            % 设置粗体系列
            \bfseries
            % 设置段落缩进为0
            \setlength{\parindent}{0pt}
            % 输出译文题目
            \par \zihao{-3}译文题目:  {#1}
            % 输出原文题目,引用文献中的标题
            \par \zihao{-3}原文题目: \citefield{#2}{title}
            % 输出原文来源,完整引用文献
            \par \zihao{-3}原文来源: \fullcite{#2}
            % 引用文献
            \nocite{#2}
        }
        {
            % 换行
            \newline
            % 输出摘要
            \par \textbf{摘要:}  {#3}
            % 输出关键字
            \par \textbf{关键字:} {#4}
        }
    }
}

\newcommand{\hiddenTocSucsec}{
    \changelocaltocdepth{0}
}

\newcommand{\showTocSubsec}{
    % 根据 general/layout.tex 的设置,目录深度为 2
    \changelocaltocdepth{2}
}

